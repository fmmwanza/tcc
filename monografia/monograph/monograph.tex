% 
% HOW TO COMPILE:
% 1) I used the Texmaker at Ubuntu with PDFLaTeX compiler to pdf. So basically call the quick build, and after call Bibtex and Makeindex.
% Makeindex configuration is the below mode:
% makeindex % -s %.ist -o %.nls
% 2) After, for the glossaries, execute at shell:
% makeglossaries <your_monograph_file_name>
% Call quick build one or two more times to correct the indexes.


%%%%%%%%%%%%%%%%%%%
%
% SETTING THE LATEX
%
%%%%%%%%%%%%%%%%%%%

\documentclass[tc,english,openright]{iiufrgs}

% internationalization for brazilian portuguese
%\usepackage[brazilian]{babel}

% document encoding
\usepackage[utf8]{inputenc}
%\usepackage[latin1]{inputenc}

% create landscape pages
%\usepackage{lscape}

% to include comments in body text
\usepackage{comment}

% add content to Table of Contents
\usepackage{tocbibind}

% includes bibtex bibliography
% reference: http://merkel.zoneo.net/Latex/natbib.php
%\cite{id} %\citep{id} %\citet{id}
\usepackage{natbib}

% creates an appendix page
\usepackage[toc,title]{appendix}

% text format
\usepackage[T1]{fontenc} % font
%\usepackage{verbatim} % typewriter
\usepackage{times} % Times New Roman
%\usepackage[overload]{textcase} % for the problem with altercase in sections and acronym

% typeset algorithm
%\usepackage[lined, boxed, linesnumbered]{algorithm2e}
%\newcommand{\listofalgorithmes}{\tocfile{\listalgorithmcfname}{loa}}
%\usepackage{listings}

% advanced math format
%\usepackage{amsthm} % theorem's
%\usepackage{amsfonts} % fonts
%%\usepackage{amsmath} % symbols, unnecessary with mathtools
%\usepackage{mathtools} % more symbols

% for tables
%\usepackage{array}
%\usepackage{makecell}

% for images
%\usepackage{graphicx}
%\usepackage{float}
%\usepackage{caption}
%\usepackage{subcaption}
%\usepackage{subfig}
%\usepackage{tikz}

% links definition
\usepackage{url}
\usepackage[pdftex,colorlinks=false]{hyperref}

%
% below the packages that need to stay after hyperref
%

% create an index file for glossaries
\usepackage{makeidx}
\makeindex

% create the list of abbreviations, replaced for glossaries package
%\usepackage[intoc,refpage]{nomencl}
%\renewcommand*\pagedeclaration[1]{~p.\,#1}
%\renewcommand{\nomname}{List of Abbreviation and Acronyms}
%\makenomenclature
%\nomenclature{$abbrv$}{full name}} % example
% makeindex %.nlo -s %.ist -o %.nls %% needed in the editor settings

% create the list of abbreviations and glossaries
\usepackage[acronym,shortcuts,toc]{glossaries}
\makeglossaries
%% examples
%\newacronym{ida}{abbrv}{full name}
%\newglossaryentry{id}{name={abbrv},description={description},first={full name (abbrv)}}
%\gls{id} % print the item glossary id
%\glsreset{ida} % restarts the acronym ida full name
%\gls{ida} % print the acronym ida

%%%%%%%%%%%%%%%%%%%%%%%%%%%%%%%%%%%%%%%%%
%
% BEGINNING THE  WRITING OF THE MONOGRAPH
%
%%%%%%%%%%%%%%%%%%%%%%%%%%%%%%%%%%%%%%%%%

% update nominata data
\renewcommand{\nominata}{
       UNIVERSIDADE FEDERAL DO RIO GRANDE DO SUL\\
       Reitor: Prof.~Carlos Alexandre Netto\\
       Vice-Reitor: Prof.~Rui Vicente Oppermann\\
       Pr\'{o}-Reitor de Gradua\c{c}\~{a}o: Prof.~S\'{e}rgio Roberto Kieling Franco\\
       Diretor do Instituto de Inform\'{a}tica: Prof.~Lu\'{i}s da Cunha Lamb\\
       Coordenador do CIC: Prof.~Raul Fernando Weber\\
       Bibliotec\'{a}ria-chefe do Instituto de Inform\'{a}tica: Beatriz Regina Bastos Haro
}

\title{Your english title}
\author{Francis Mwanza}

\location{city}{state acronym}
\date{July 20th,}{2013}

\advisor[Profa.~Dra.]{second name}{Advisor first Name}

% an set of keywords for CIP
\keyword{keyword 1}
\keyword{keyword 2}
\keyword{keyword 3}
\keyword{keyword 4}
\keyword{keyword 5}

% define space for paragraph
\setlength{\parskip}{1.9mm}
%\gdef\citet#1{\citeauthor*{#1} \citeyearpar{#1}}

\begin{document}

%\sloppy
\maketitle % ok

% inclusion of version control
%%%%%%%%%%%%%%%%%%%%%%%%%%%%%%
%
% INCLUSION OF VERSION CONTROL
%
%%%%%%%%%%%%%%%%%%%%%%%%%%%%%%

%create a version of the document for control
\begin{center}
	\catcode`\=9
	Draft 1.0 ~~\today  \\
	\catcode`\=3
	\texttt{yourUser@inf.ufrgs.br}\\
	http://inf.ufrgs.br/$\sim$yourUser
\end{center}

% inclusion of the quote from a personality
%%%%%%%%%%%%%%%%%%%%%%%%%%%%%%%%%%%%%%%
%
% INCLUSION OF QUOTE FROM A PERSONALITY
%
%%%%%%%%%%%%%%%%%%%%%%%%%%%%%%%%%%%%%%%

\chapter*{}
\null
\vfill
\begin{quote}
Space to include an citation for important people, or any small text. It is not necessary.
\end{quote} % ok

% inclusion of acknowledgments
%%%%%%%%%%%%%%%%%%%%%%%%%%%%%%
%
% INCLUSION OF ACKNOWLEDGMENTS
%
%%%%%%%%%%%%%%%%%%%%%%%%%%%%%%

\chapter*{Acknowledgments}
The contents...

\tableofcontents
\listoffigures
\listoftables
%\listofalgorithmes
%\printnomenclature
\printglossary[type=\acronymtype]

% inclusion of the abstracts
%%%%%%%%%%%%%%%%%%%%%%%%%%%%%%%
%
% INCLUSION OF ENGLISH ABSTRACT
%
%%%%%%%%%%%%%%%%%%%%%%%%%%%%%%%

\addcontentsline{toc}{chapter}{Abstract}

\begin{abstract}
The contents...

\end{abstract}

%%%%%%%%%%%%%%%%%%%%%%%%%%%%%%%%%%
%
% INCLUSION OF PORTUGUESE ABSTRACT
%
%%%%%%%%%%%%%%%%%%%%%%%%%%%%%%%%%%

\selectlanguage{brazilian}

\addcontentsline{toc}{chapter}{Resumo}

\begin{englishabstract}{Titulo em portugues}{lista de palavras chave em portugues}
O conteúdo...ku wg sdknfsdf

\end{englishabstract}

\selectlanguage{english}


% inclusion of the chapters
%%%%%%%%%%%%%%%%%%%%%%%%%%%
%
% INCLUSION OF INTRODUCTION
%
%%%%%%%%%%%%%%%%%%%%%%%%%%%

\chapter{Introduction} % 2 pages
\label{chap:introduction}

%
% Motivation
%

\section{Motivation}
The contents...
Citation \cite{bk:sbc2007}
Citation \citep{ds:aco2004}
Citation \citet{dmc:as1996}
Citation \citep{bbek:simul2011}
Citation \citep{rossum:lang1995}
Citation \citep{silva:ti2005}
Citation \citep{oliveira:master2005}
Citation \citep{Joe:COFFEE2000}
Citation \citep{COFFEE:2000}

\newacronym{obna}{OBN}{One Big Name}
\newglossaryentry{obn}{name={OBN},description={One Big Name description},first={One Big Name (OBN)}}

Glossary term: \gls{obn} % print the item glossary obn

Acronym term: \gls{obna}
Acronym term after apparition: \gls{obna}
\glsreset{obna} % restarts the acronym obna full name
Acronym term reset: \gls{obna} % print the acronym obna

%
% Goals
%
% Maybe can be included directly at introduction
\section{Goals}
The contents...

%
% Contribution
%
% Maybe is better to include this at results chapter
\section{Contribution}
The contents...

%
% Strutucture of this work
%

\section{Structure of this work}
The contents...
%%%%%%%%%%%%%%%%%%%%%%%%%%%%%%%
%
% INCLUSION OF CONCEPTUAL BASIS
%
%%%%%%%%%%%%%%%%%%%%%%%%%%%%%%%

\chapter{Conceptual Basis} % 1 to 1-1/2 pages for section
\label{chap:conceptual_basis}

\section{Section}
The contents...

\subsection{Subsection}
The contents...

\paragraph{Paragraph}
The contents...

\subparagraph{Sub paragraph}
The contents...


%
% FINAL REMARKS
%

\section{Final remarks}
The contents...
%%%%%%%%%%%%%%%%%%%%%%%%%%
%
% INCLUSION OF METHODOLOGY
%
%%%%%%%%%%%%%%%%%%%%%%%%%%

\chapter{Methodology} % 8-10 pages
\label{chap:methodology}

The contents...

\section{Section}
\label{sec:meth:section}

\section{Subsection}
The contents...

%
% FINAL REMARKS
%

\section{Final remarks}
%%%%%%%%%%%%%%%%%%%%%%%%%%%%%%%%%%%%%%
%
% INCLUSION OF EXPERIMENTAL EVALUATION
%
%%%%%%%%%%%%%%%%%%%%%%%%%%%%%%%%%%%%%%

\chapter{Experimental Evaluation} % 8-10 pages for the involved experiments
\label{chap:experimental_evaluation}

The contents...

\section{Section}
The contents...

%\begin{algorithm}
%\caption{Algorithm Name}\label{alg:AlgorithmName}
%\end{algorithm}


%
% FINAL REMARKS
%

\section{Final remarks}
%%%%%%%%%%%%%%%%%%%%%%
%
% INCLUSION OF RESULTS
%
%%%%%%%%%%%%%%%%%%%%%%

\chapter{Results} % 3 pages
\label{chap:results}

The contents...

%
% CONCLUDING REMARKS
%

\section{Concluding Remarks}
The contents...
%%%%%%%%%%%%%%%%%%%%%%%%%%%%%%%%%%%%%%%%%%
%
% INCLUSION OF CONCLUSIONS AND FUTURE WORK
%
%%%%%%%%%%%%%%%%%%%%%%%%%%%%%%%%%%%%%%%%%%

\chapter{Conclusions and Future Work}
\label{chap:conclusions_and_future_work}

The contents...

%
% FUTURE WORK
%

\section{Future Work}
The contents...

% inclusion of bibliography
% DBLP-bibtex: http://www.dblp.org/search/index.php
\bibliography{bibliography}
\bibliographystyle{plainnat}

% inclusion of the appendices
%%%%%%%%%%%%%%%%%%%%%%%%%%
%
% INCLUSION OF APPENDICES
%
%%%%%%%%%%%%%%%%%%%%%%%%%%

\begin{appendices}

\chapter{}

\section{Some Appendix A}
The contents...

\section{Some Appendix B}
The contents...

\end{appendices}

% inclusion of the glossary
% linking acronym with glossary: http://tex.stackexchange.com/questions/8946/how-to-combine-acronym-and-glossary
\printglossary

\end{document}
